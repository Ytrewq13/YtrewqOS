\documentclass{article}

\usepackage[margin=0.55in,bottom=0.75in]{geometry}
\usepackage{titlesec}
\usepackage{datetime}

\usepackage{graphicx}

\usepackage{amssymb}
\usepackage{amsmath}

% Format footnotes (bottom of page, reset counter at every page break, use
% symbols instead of numbers)
\usepackage[bottom,perpage,symbol*]{footmisc}

\usepackage{csquotes}
\usepackage[british]{babel}
\usepackage[backend=biber,urldate=iso,date=iso,seconds=true]{biblatex}
\addbibresource{refs.bib}

\usepackage{xcolor}

\usepackage{pgfgantt}

\usepackage{hyperref}

\usepackage{subfiles}

\newdateformat{mydate}{\dayofweekname{\THEDAY}{\THEMONTH}{\THEYEAR}, \ordinal{DAY} \monthname[\THEMONTH], \THEYEAR}

\renewcommand{\baselinestretch}{0.85}

\begin{document}
\begin{titlepage}
    % Custom cover page
    \begin{center}
        \null\mbox{}\vfill

        \vspace*{1cm}

        \huge
        \textbf{Operating System for the Raspberry Pi}

        \vspace{0.5cm}
        \Large
        A Unix-like Kernel, Shell, and Coreutils for the 64-bit Raspberry Pi 3b

        \Large

        \vspace{1.5cm}

        \textbf{Sam Whitehead} - 14325283

        \texttt{psysrw@nottingham.ac.uk}

        Msci Computer Science

        \vfill

        COMP4029: Individual Programming Project\\
        Project Supervisor: Steve Bagley\\
        University of Nottingham

        \vfill\null
    \end{center}
    \thispagestyle{empty}
\end{titlepage}

\section{Aims and Objectives}
% FIXME: reword this section
I am aiming to produce a Unix-like Operating System which is capable of being
used as a general purpose hobbyist OS. The objectives of the project are as
follows.

\subsection*{OS components/features I expect to complete}
\begin{itemize}
    \item Microkernel with UART serial output and graphics driver for the Pi's
        HDMI video output. User interaction will use the display and read from
        the keyboard.
    \item Shell (hopefully POSIX-compliant) with scripting capabilities.
        Depending on time constraints, this may be a port of
        Dash~\cite{dash-shell}.
    \item Multi-process system allowing multiple processes to time-share the
        CPU with a process scheduler.
    \item An implementation of the System V ABI for executable files
        (statically compiled ELF files).
    \item Suite of core utilities (like those provided by GNU) - these may be
        partially or totally ported from the GNU
        coreutils~\cite{gnu-coreutils}.
    \item An init program for starting the OS's essentials and for starting the
        root\footnote{Here, ``root'' means PID 0, not a root user (although if
        users are implemented, it would be both).} shell.
    \item A service manager for starting services at boot and ensuring that
        they remain running, restarting them if they fail. This may be bundled
        with the init system, like with systemd on most Linux distributions, or
        may be a separate program.
\end{itemize}

\subsection*{Nice-to-haves}
\begin{itemize}
    \item A threads system with multithreading support (multiple threads owned
        by one process).
    \item A multi-user system with basic passwords and access control on files
        (file/directory ownership).
    \item Runs on hardware (not just a VM/emulator).
\end{itemize}


\section{Work Plan}
% FIXME: reword this section
\subsection*{Tasks}
Tasks are the individual components of the project, separated into the
components of the Kernel and Shell, the miscellaneous parts, and the documents
I will need to produce for the project. These tasks are enumerated in
\autoref{tab:work-plan} and then assembled into a Gantt chart in
\autoref{fig:gantt-chart}.
\begin{table}[tbp]
\begin{center}
\begin{tabular}{|l|r|}
    \hline
    Task & Duration (weeks)\\
    \hline \textbf{Kernel} &\\
    Bootloader & 1\\
    Graphics driver & 2\\
    Syscalls & 6\\
    Filesystem & 3\\
    Statically linked ELF Loader & 2\\
    Memory management & 4\\
    Process control & 8\\
    Init system and service manager & 5\\
    Threads and multithreading & 4\\
    \hline \textbf{Shell} &\\
    Debug shell & 2\\
    \texttt{pwd}, \texttt{cd}, \texttt{ls}, \texttt{stat}, etc. & 2\\
    \texttt{export}, variables, \texttt{set}, etc. & 2\\
    \texttt{if}, \texttt{while}, \texttt{for}, \texttt{case}, globs, etc. & 3\\
    Command substitution & 4\\
    IO pipes and output redirection & 1\\
    \hline \textbf{Misc parts} &\\
    Setup cross-compiler and build environment & 1\\
    IO and strings libraries & 2\\
    libc & 8\\
    \texttt{cat}, \texttt{head}, \texttt{less}, etc. & 2\\
    \texttt{roff} & 1\\
    \texttt{man} & 1\\
    \hline \textbf{Documents} &\\
    Interrim report & 1\\
    Documentation & 5\\
    Dissertation & 7\\
    \hline
\end{tabular}
\caption{The plan of work for the project, including how many weeks I think
each component will take.}
\label{tab:work-plan}
\end{center}
\end{table}

\subsection*{Work Packages}
A ``Work Package'' is a collection of tasks with a well-defined deliverable.
Not every task will be part of a work package, and some tasks will be work
packages on their own. \autoref{tab:work-packages} contains all of the work
packages for this project.

\begin{table}[tbp]
\begin{center}
\begin{tabular}{|p{45mm}|p{125mm}|}
    \hline
    \textbf{Work package} & \textbf{Description of the deliverable} \\
    \hline
    Debug shell using the graphics driver &
    A debug shell which is printed on the Pi's display and which uses keyboard
    input. The debug shell should accept several commands for debugging the
    processor status (including, but not limited to, printing the contents of
    memory, writing to memory, and jumping to a given memory address).
    \\ \hline
    A basic kernel which can load (from disk) and jump into a compiled ELF
    program. &
    A program that can load a statically compiled ELF executable binary into
    memory and then jump to its entry point. This requires a filesystem library
    to be implemented, and also includes the ELF loader task.
    \\ \hline
    Scheduling algorithm &
    A scheduling algorithm with support for multiple processes running on the
    system concurrently. This will include an implementation of
    \texttt{fork(3)} or a similar function in order to spawn new processes and
    \texttt{execve(2)} to replace the current program with a new process.
    \\ \hline
    Init system and service manager &
    An init system which is run at boot and starts all necessary parts of the
    OS and then spawns the root shell. The service manager ensures that all
    desired daemons and processes are started at the correct times and remain
    running, restarting them if they die.
    \\ \hline
    A shell &
    A (POSIX-compliant) shell \textbf{or} a port of a simple shell like
    Dash\cite{dash-shell}. If I decide that implementing my own
    POSIX-compliant shell is too large of a task, then I will port the source
    code for Dash to run on my OS, implementing the required syscalls to get it
    working. This shell will be the main way users interact with the OS.
    \\ \hline
    Documentation &
    This package contains several parts. I will create documentation for how
    each part of the OS works, and a guide for how new programmers can get
    started developing programs for the system. The documentation will be
    completed in stages as I develop the operating system, and I will go back
    to keep it up-to-date as I make changes to components and add new
    components.
    \\ \hline
    Multithreading support and multi-core scheduling &
    The Pi 3 has 4 CPU cores. This work package will enable users to take
    advantage of the additional processing power of the other 3 cores for their
    programs. This includes a threading library similar to \texttt{pthread} on
    Linux systems, and improvements to the scheduler so that it can give
    different CPU cores to multiple threads owned by the same process.
    \textbf{This work package is optional}.
    \\ \hline
\end{tabular}
\caption{The work packages of the project and their deliverables.}
\label{tab:work-packages}
\end{center}
\end{table}

\subfile{gantt.tex}

% Mark each of our references as cited without actually citing them.
\nocite{design-of-unix-os}
\nocite{osdev-wiki}
\nocite{unix-prog-env}
% Print the bibliography
\printbibliography
\end{document}
